\documentclass[a4paper]{article}
\usepackage[fleqn]{amsmath}
\usepackage{graphicx}
%\usepackage{times}
\usepackage[framed,numbered,autolinebreaks,useliterate]{mcode}
\usepackage{listing}
\usepackage[small,compact]{titlesec}
\usepackage[utf8]{inputenc}

\usepackage{biblatex}
\bibliography{Response}

\usepackage[paper=a4paper,
            %includefoot, % Uncomment to put page number above margin
            marginparwidth=30.5mm,    % Length of section titles
            marginparsep=1.5mm,       % Space between titles and text
            margin=15mm,              % 25mm margins
            includemp]{geometry}


\newcommand{\makeheading}[2]%
        {\hspace*{-\marginparsep minus \marginparwidth}%
         \begin{minipage}[t]{\textwidth\marginparwidth\marginparsep}%
           {\large \bfseries #1}\\{#2}\\[-0.15\baselineskip]%
                 \rule{\columnwidth}{1pt}%
         \end{minipage}}

\newlength{\figurewidth}
\setlength{\figurewidth}{500px}


\begin{document}
\makeheading{Gautebøye, rev1, Instrument response}{Gaute Hope
(gaute.hope@student.uib.no)}

\section{Hydrophone decoupling}
\begin{figure}[h]
  \begin{center}
  \includegraphics[width=400px]{Hydrophone_decoupling.png}
\end{center}
  \caption{Hydrophone decoupling circuit}
  \label{fig:hydrophone_decoupling}
\end{figure}

The figure above (\ref{fig:hydrophone_decoupling}) shows an equivalent
circuit of the deployed input decoupling to the hydrophone.

\begin{align*}
  C_1 &= 47 \mu F \\
  C_2 &= 380 pF \\
  R_1 &= 58.33 k \Omega \\
  R_2 &= 41.67 k \Omega
\end{align*}

\paragraph{}The circuit was based on the suggested decoupling in figure
\ref{fig:suggested_decoupling} seen below, it was designed to have the
same properties, but also low-pass the signal at cutoff frequency
$f_{LP_1}$. The original design already decouples, and high-passes, the
signal at frequency $f_{HP}$.

\begin{figure}[h!]
  \begin{center}
    \includegraphics[width=200px]{AQ-18-decoupling-and-wiring.jpg}
  \end{center}
  \caption{AQ-18 suggested decoupling and wiring
    \cite{aq-18_decoupling}.}
  \label{fig:suggested_decoupling}
\end{figure}

\section{Derivation of transfer function}

$i_1$ and $i_2$ signify the current loops passing through respectively left
and right loop. Using Kirchoff's equation:

\begin{align}
  \label{eqn:currents_1}
  V_s &= i_1 (Z_{C_1} + Z_{R_1}) + (i_1 - i_2) \cdot Z_{R_2} \\
  0   &= (i_2 - i_1) \cdot Z_{R_2} + i_2 \cdot Z_{C_2}
  \label{eqn:currents_2}
\end{align}

$V_s$ is the signal source, the hydrophone. See separate data sheet for
hydrophone frequency response. $V_o$, output, is measured at the
terminals $T_+$ and $T_-$.

\begin{align}
  \label{eqn:vo}
  V_o &= i_2 \cdot Z_{C_2} \\
  H(s) &= \frac{V_o}{V_s}, s = i\omega
\end{align}

Solving for $H(s)$ gives:

\begin{equation}
  H(s) = \frac{45115.36 s}
  { (s + 1.083 \times 10^5)
    (s + 0.2128) }
  \label{eqn:transferfunction}
\end{equation}

A bode plot of the transfer function (instrument response) is shown in
figure \ref{fig:bodeplot} below.
\begin{figure}[h!]
  \begin{center}
    \includegraphics[width=400px]{bodeplot.png}
  \end{center}
  \caption{Bode plot of transfer function, showing peak response at
  $33.9$ Hz.}
  \label{fig:bodeplot}
\end{figure}

Equation \ref{eqn:transferfunction} can be solved by using the MATLAB
script response.m (listing \ref{code:response_m}), the Bode plot in figure \ref{fig:bodeplot} was created with the same script.

\subsection{Analysis}
The initial frequency band of interest reaches
from periods of 20 s (0.05 Hz) to approximately $100$ Hz. The analog input part as
well as the configuration of the analog-digital converter has been
configured to as best and simple as possible to record the signal
within this frequency band.

\subsubsection{Decoupling and analog high pass filter}
The transfer function is a bandpass filter with
the suggested decoupling creating a high-pass filter with corner
frequency at $f_{HP} = 0.0339$ Hz or a period of $29.4985$ s.

\subsubsection{Low pass filter and operational amplifier}

To avoid aliasing due to the speed of the opamps (operational
amplifiers) in the signal path, first the OPA629 (see \cite{opa629_ds} and
figure \ref{fig:hydrophone_decoupling}) acting as buffer with
unity gain and then the OPA1632 \cite{opa1632_ds} at the input terminals
of the ADS1282EVM \cite{ads1282evm_ds}.
An additional low pass stage is added with a corner frequency of
$f_{LP_1} = 17.189 $ kHz.

\subsubsection{Stability}
The system, $H(s)$ from equation \eqref{eqn:transferfunction}, is
passive and has all
its poles in the left half plane and is stable.

\subsubsection{Analog-Digital Converter (ADC)}
The ADS1282EVM equips its ADS1282 with a crystal with the recommended clock frequency of
$f_{clk} = 4.096$ Mhz \cite{ads1282evm_ds}. This results in a modulator
output speed of $ f_s = f_{clk}/4 = 1.024$ Mhz \cite{ads1282_ds}. The
Nyquist frequency is then: $f_{Nq} = 1.024 \times 10^6 / 2 = 512 $ kHz.

\subsubsection{Anti-alias filter}
The 10 nF capacitor used on the ADS1282EVM
\cite{ads1282evm_ds}, results in a low-pass filter with an upper corner frequency of $f_{LP_2} =
\frac{1}{2 \pi \times 600 \times 10 \times 10^{-9}} = 26525.82 $ Hz
\footnote{Equation (3), p.15, \cite{ads1282_ds}; \citetitle{ads1282_ds}}.

$f_{LP_2}$ is thus well below $f_{Nq}$ and any signal component above $f_{LP_2}$ should
be almost completely damped before it reaches $f_{Nq}$, resulting in no
or very small aliasing effects.

This equation represents the upper low pass filter:
\begin{equation}
  H_{UL}(s) = \frac{1 \div 600 \times 10 \times 10^{-9}}
                   {(s + 1 \div 600 \times 10 \times 10^{-9})}
            = \frac{1.667 \times 10^5}{(s + 1.667 \times 10^5)}
  \label{eqn:transfer_upper_lowpass}
\end{equation}

The complete analog transfer function is then:
\begin{equation}
  H_{analog} = \frac{7.5192 \times 10^9}
                { (s + 1.667 \times 10^5)
                  (s + 1.083 \times 10^5)
                  (s + 0.2128) }
  \label{eqn:transfer_analog}
\end{equation}

With the following frequency response:
\begin{figure}[h!]
  \begin{center}
    \includegraphics[width=400px]{bode_complete_analog.png}
  \end{center}
  \caption{Frequency response of complete analog system, equation
    \eqref{eqn:transfer_analog}.}
  \label{fig:bode_complete_analog}
\end{figure}

\subsubsection{Digital filter}
Further the ADS1282 is configured to digitally filter the decimation
output so that the final output signal of 250 Hz contains no alias. The
built-in sinc and FIR filters are activated, see the ADS1282 datasheet
\cite{ads1282_ds} for details. Further, the high pass filter has been
set to filter at $f_{HP} = 0.05\ \text{Hz} = 20s$.

\paragraph{}The
ADS1282 can be configured for an sample rate of as high as 4000 Hz
\cite{ads1282_ds}. This would be possible without changes in the analog interface, although it would come at an cost of resolution.

\subsubsection{Output} The output is scaled by $-7.6\ \text{dB} =
0.4169 \frac{V}{V}$ which is to scale the input from $\pm 6V$ to $\pm 2.5V$, $0.4169
\approx 5 / 12 = 0.4167$, which is the input range of the ADS1282EVM
(\cite{ads1282evm_ds} and \cite{ads1282_ds}) in bipolar mode. The
slightly inaccurate scaling is due to available resistor values.

\newpage
\printbibliography

\newpage
\section{Attachments}
\subsection{response.m}
\lstset{caption={response.m, MATLAB code for calculating instrument
response},label=code:response_m}
\lstinputlisting{response.m}

\end{document}

