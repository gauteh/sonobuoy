\documentclass[a4paper]{article}
\usepackage[fleqn]{amsmath}
\usepackage{graphicx}
%\usepackage{times}
\usepackage[framed,numbered,autolinebreaks,useliterate]{mcode}
\usepackage{listing}
\usepackage[small,compact]{titlesec}
\usepackage[utf8]{inputenc}

\usepackage{biblatex}
\bibliography{../Documentation}

\usepackage[paper=a4paper,
            includefoot, % Uncomment to put page number above margin
            marginparwidth=30.5mm,    % Length of section titles
            marginparsep=1.5mm,       % Space between titles and text
            margin=10mm,              % 25mm margins
            includemp]{geometry}

%\setlength{\oddsidemargin}{10mm}
%\setlength{\evensidemargin}{10mm}
\usepackage{fullpage}

\usepackage{multicol}
\usepackage{caption}

\newcommand{\makeheading}[2]%
        {\hspace*{-\marginparsep minus \marginparwidth}%
         \begin{minipage}[t]{\textwidth\marginparwidth\marginparsep}%
           {\large \bfseries #1}\\{#2}\\[-0.15\baselineskip]%
                 \rule{\columnwidth}{1pt}%
         \end{minipage}}

\newlength{\figurewidth}
\setlength{\figurewidth}{500px}


\begin{document}
\makeheading{Gautebøye - Operators manual}{Gaute Hope
(gaute.hope@student.uib.no), 02.09.2012, Revision 1}

\vspace{2em}
\section*{Introduction}
This document describes how to install and operate a network of
Gautebøyer.

\vspace{2em}
\section{Installation}

\subsection{Central logging point: Zero}
  Zero is the central logging point and is started by running the script
  'zero/zero.py' from the command line.


  \subsubsection{ZeroNode}
    For the central logging point you need a computer connected to an
    RF200 node through USB with the snap script 'zero/snap/zeronode.py'
    installed.

\subsection{Buoy}

\section{Monitoring}



\vspace{5em}
%\printbibliography

\end{document}

